\documentclass[10pt,fleqn,xcolor=dvipsnames]{beamer}
\beamertemplatenavigationsymbolsempty
\usetheme{Boadilla}
\DeclareGraphicsExtensions{.jpg,.eps,.png,.pdf,.mps,.gif}
\usepackage[latin1]{inputenc}
\usepackage[T1]{fontenc}
\usepackage[english]{babel}
\usepackage{pgf,pgfarrows,pgfnodes,pgfautomata,pgfheaps}
\usepackage{amsmath}
\usepackage{amsfonts}
\usepackage{graphicx}
\usepackage{color}
\usepackage{lmodern}
\usepackage[3D]{movie15}
\usepackage{algorithm}
\usepackage{algpseudocode}
\usepackage{url}
\usepackage{placeins}
\usepackage{listings}

\colorlet{grey}{gray!40}
\definecolor{PG}{rgb}{0.0, 0.27, 0.13}
%%%%%%%%%%%%%%%%%%%%%%%%%%%%%%%%%%%%%%%%%%%%%%%%%%%%%%%%%%%%%%%%%%%%%%%%%%%%%%%%%%%%%%%
% title page definition %%%%%%%%%%%%%%%%%%%%%%%%%%%%%%%%%%%%%%%%%%%%%%%%%%%%%%%%%%%%%%%
%%%%%%%%%%%%%%%%%%%%%%%%%%%%%%%%%%%%%%%%%%%%%%%%%%%%%%%%%%%%%%%%%%%%%%%%%%%%%%%%%%%%%%%
\setbeamercovered{dynamic}
\setbeamerfont{author}{family=\rmfamily}
\author[M. Beccuti,   R. J. P. Bonnal
and  R. Calogero]{\large{\textbf{M. Beccuti,  R. J. P. Bonnal
 and R. Calogero}}\\[15pt]\emph{Universit\`{a} degli Studi di Torino \\  Istituto Nazionale di Genetica Molecolare Romeo ed Enrica Invernizzi
}\\~\\~ \textbf{ELIXIR-IIB Training Platform}~\\\tiny(Second day)}
\title[BITS 2018]{\Large{\textbf{Docker and Reproducibility}}}
%\institute{\normalsize{} \\
%\footnotesize{Boston Park Plaza Hotel, Boston, Massachusetts USA}}
\date[June 2019]{June 2019}
\titlegraphic{
\includegraphics[height=2cm]{Figure/LogoTorino}
\includegraphics[height=2cm]{Figure/LogoINGM}
}


% table of contents depth
\setcounter{tocdepth}{1}
\begin{document}
\usebackgroundtemplate{\includegraphics[width=140mm]{Figure/BackgroundTorino}}
\setcounter{tocdepth}{5}
%#Frame 1 title
\frame[plain]{\titlepage}


\frame{
  \frametitle{Outline}  \vspace{0.3cm}
    \begin{block}{Training aims:}
\begin{itemize}
\item Create and run  a docker image; 
\item Develop a  docker service;
\item ...

\end{itemize}    



\end{block}
 \vspace{0.5cm}
\begin{enumerate}
\item A short introduction  recalling the concepts described in the first day\vspace{0.2cm}
\begin{itemize}
\item Virtualization: Virtual Machines and containers; \vspace{0.2cm}
\item Containers in Linux: Docker project; \vspace{0.2cm}
\end{itemize}

\item A simple example: how to embed an application  in docker image; \vspace{0.2cm}
\item Create a docker image for FastQC tool;\vspace{0.2cm}
\item A web server  using docker;\vspace{0.2cm}
\item Deploy services on a cluster using docker swarm.
\end{enumerate}
  }
  
  
  \frame{
  \frametitle{What are \emph{containers} and \emph{Virtual Machines}?}
  \begin{columns}[T] % align columns
  \begin{column}{.80\textwidth}
  	\begin{itemize}
   		\item Containers and Virtual Machines (VM) are similar in their goals: 
\begin{enumerate}   		
   		\item \emph{\color{NavyBlue}to provide analysis \textbf{portability} isolating an application  into a self-contained unit that can run anywhere};\vspace{0.3cm}
   		\item \emph{\color{NavyBlue}to provide analysis \textbf{reproducibility} freezing the version of  tools and libraries used.}
   		\end{enumerate}
   		\vspace{0.3cm}
   		\item  They remove the need for physical hardware, allowing for more efficient use of computing resources, both in terms of energy consumption and cost effectiveness;\vspace{0.3cm}
   		\item 	The main difference between containers and VMs is in their architectural approach.
   		\end{itemize}\vspace{0.3cm}
   \end{column}%
   \hfill%
   \begin{column}{0.20\textwidth}
     	\begin{center}
  			\includegraphics[width=1.15\columnwidth]{./Figure/DockerVSVM}\\
  		\end{center}
   \end{column}%
   \end{columns}
  } 



 \frame{
  \frametitle {Virtual Machines}
  
  \begin{columns}[T] % align columns
  \begin{column}{.55\textwidth}
  	\begin{itemize}
 		\item A VM is essentially an emulation of a real computer (i.e.\emph{ \color{NavyBlue} guest machine}) that executes programs like a real computer;\vspace{0.3cm}
 
 		\item  VMs run on top of a physical machine (i.e.\emph{\color{NavyBlue} host machine}) using a \emph{\color{NavyBlue}hypervisor};\vspace{0.3cm}
 
 		\item A {\color{NavyBlue}hypervisor} is a piece of software, firmware, or hardware;\vspace{0.3cm}
 		
 		\item A {\color{NavyBlue}guest machine} contains:
		
		\begin{itemize}	 		
 		
 		 \item both the application and whatever it needs to run that application (e.g. system binaries and libraries). 
 	
 	    \item a virtualized hardware stack including virtualized network adapters, storage, CPU \dots.
 	
 		\end{itemize}
 		
      	\end{itemize}
   \end{column}%
   \hfill%
   \begin{column}{0.45\textwidth}
     	\begin{center}
  			\includegraphics[width=1.00\columnwidth]{./Figure/VM}\\
  		\end{center}
   \end{column}%
   \end{columns}
  } 
  
  
  
  
  \frame{
  \frametitle {Containers}
  
  \begin{columns}[T] % align columns
  \begin{column}{.55\textwidth}
  	\begin{itemize}
 		\item Unlike a VM which provides hardware virtualization, a container provides operating-system-level virtualization by abstracting the \emph{\color{NavyBlue} user space};\vspace{0.3cm}
 
 		\item The main difference between containers and VMs is that containers \textbf{share} the host system's kernel with other containers.\vspace{0.3cm}
 
      	\end{itemize}
   \end{column}%
   \hfill%
   \begin{column}{0.45\textwidth}
     	\begin{center}
  			\includegraphics[width=1.05\columnwidth]{./Figure/Container}\\
  		\end{center}
   \end{column}%
   \end{columns}
  } 
  

 \frame{
  \frametitle {Containers VS VM}
  
  \begin{columns}[T] % align columns
  \begin{column}{.50\textwidth}
		\begin{center}
  			\includegraphics[width=1.00\columnwidth]{./Figure/Container}\\
  		\end{center}
   \end{column}%
   \hfill%
   \begin{column}{0.50\textwidth}
     	\begin{center}
  			\includegraphics[width=1.00\columnwidth]{./Figure/VM}\\
  		\end{center}
   \end{column}%
   \end{columns}
  } 

\frame{
  \frametitle {Containers VS VM}
 	\begin{center}
  			\includegraphics[width=0.90\columnwidth]{./Figure/ContainersVSVM}\\
  		\end{center}
  } 

\frame{
\frametitle{Docker container, VM and real server: a comparison}
In [1] a comparison among physical server, KVM, and Docker is reported.

     	\begin{center}
  			\includegraphics[width=0.90\columnwidth]{./Figure/benchmark}
  		\end{center}  
{\tiny
[1] W. Felter, A. Ferreira, R. Rajamony and J. Rubio, \emph{\textbf{An updated performance comparison of virtual machines and Linux containers}}, 2015 IEEE International Symposium on Performance Analysis of Systems and Software (ISPASS), Philadelphia, PA, 2015, pp. 171-172.

}
} 



\frame{
\frametitle{Computational Reproducibility Stack}

       	\begin{center}
  			\includegraphics[width=0.9\columnwidth]{./Figure/stack}\\
  		\end{center}
}

 \frame{
  \frametitle {}
  \centerline{\Huge \color{NavyBlue} \textbf{\emph{Docker project}}}
       	\begin{center}
  			\includegraphics[width=0.8\columnwidth]{./Figure/basic}\\
  		\end{center}
}

  
  
 \frame{
  \frametitle {Docker project}
  
  
  
  \begin{columns}[T] % align columns
  \begin{column}{.55\textwidth}
  	\begin{itemize}
 		\item Docker is an open-source project based on Linux containers.
 	\vspace{0.3cm}
 		\item Others Linux container technologies include Solaris Zones, BSD jails, and LXC, which have been around for many years.
 		 	\vspace{0.3cm}	
      	\end{itemize}
   \end{column}%
   \hfill%
   \begin{column}{0.45\textwidth}
     	\begin{center}
  			\includegraphics[width=1.05\columnwidth]{./Figure/Docker}\\
  		\end{center}
   \end{column}%
   \end{columns}
   \centerline{\Large \color{NavyBlue} \textbf{Why to use Docker??}}
  } 
  
  
  

 \frame{
  \frametitle {Why to use Docker?}
  
    \begin{columns}[T] % align columns
  \begin{column}{.65\textwidth}
  	\begin{itemize}
	\item  \textbf{\color{NavyBlue} Ease of use:} Docker has made it much easier for anyone to take advantage of containers in order to quickly build and test portable applications; \vspace{0.3cm}	
	
	\item  \textbf{\color{NavyBlue} Speed:} Docker containers are very lightweight and fast.
	\vspace{0.3cm}	
	\item \textbf{\color{NavyBlue}Docker Hub:} Docker users also benefit from the increasingly rich repository of Docker Hub, which you can think of as an "app store for Docker images"; 
	\vspace{0.3cm}	
	\item \textbf{\color{NavyBlue}Modularity and Scalability:} Docker makes it easy to break out your application's functionality into individual containers.
  	\end{itemize}
   \end{column}%
   \hfill%
   \begin{column}{0.35\textwidth}
     	\begin{center}
  			\includegraphics[width=1.05\columnwidth]{./Figure/Docker}\\
  		\end{center}
   \end{column}%
   \end{columns}
  } 
    
  

  \frame{
\frametitle{Docker basics} 
\vspace{0.2cm}
\begin{itemize}
\item \textbf{\color{NavyBlue}\emph{Image}} is an executable package that includes everything needed to run an application (i.e. its code,  libraries, environment variables, and configuration files);\vspace{0.2cm}
\item \textbf{\color{NavyBlue}\emph{Container}} is a run-time instance of an image;\vspace{0.2cm}
\item \textbf{\color{NavyBlue}\emph{Volume}} is used to  share files from the host machine to containers.
\end{itemize}
\vspace{0.2cm}
  \begin{center}
  			\includegraphics[width=0.55\columnwidth]{./Figure/exec}\\
  		\end{center} 

}

 \frame{
  \frametitle {Docker schema}
  
  
  
  \begin{columns}[T] % align columns
  \begin{column}{.55\textwidth}
  	\begin{itemize} 		
  	\item \textbf{\color{NavyBlue}Docker client:} provides an interface for users;
 	\vspace{0.2cm}
 		\item \textbf{\color{NavyBlue}Docker Host:} executes the commands sent to the Docker Client;\vspace{0.2cm}

 		\item \textbf{\color{NavyBlue}Docker Hub:} remove repository storing docker's images.
      	\end{itemize}
   \end{column}%
   \hfill%
   \begin{column}{0.45\textwidth}
     	\begin{center}
  			\includegraphics[width=0.95\columnwidth]{./Figure/DockerSchema}\\
  		\end{center}
   \end{column}%
   \end{columns}
  } 
  
  
  
  
   \frame{
  \frametitle {}
  \vspace{1.5cm}
  \centerline{\Huge \color{NavyBlue} \textbf{\emph{Create a docker image for }}}
  \centerline{\Huge \color{NavyBlue} \textbf{\emph{FastQC tool}}}
  \vspace{0.4cm}
     	\begin{center}
  			\includegraphics[width=0.60\columnwidth]{./Figure/FastQC}
  		\end{center} 

}

   \frame{
  \frametitle {Create a docker image for FastQC tool}
\vspace{0.4cm}
   \centerline{\color{NavyBlue} \textbf{\emph{FastQC in nutshell}}}\vspace{0.4cm}
   \begin{itemize}
   \item It is  quality control tool for high throughput sequence data;\vspace{0.2cm}
   \item It is written in Java;\vspace{0.2cm}
   \item It is main functions are:\vspace{0.1cm}
   \begin{itemize}
   \item Import of data from BAM, SAM or FastQ files (any variant);\vspace{0.1cm}
   \item Providing a quick overview to tell you in which areas there may be problems;\vspace{0.1cm}
   \item Summary graphs and tables to quickly assess your data;\vspace{0.1cm}
   \item Export of results to an HTML based permanent report;\vspace{0.1cm}
   \item Offline operation to allow automated generation of reports without running the interactive application.
   \end{itemize} 
   \end{itemize}     	
   \begin{center}
  			\includegraphics[width=0.15\columnwidth]{./Figure/QC}
  		\end{center} 
  }
  
 \frame{
  \frametitle {Create a docker image for FastQC tool}
   \centerline{\color{NavyBlue} \textbf{\emph{How to create FastQC image}}}\vspace{0.4cm}
  \begin{itemize}
  \item Download and update a basic image (use Fedora);\vspace{0.2cm}
  \item Download and install \textbf{\color{NavyBlue}Oracle Java} on the download image;\vspace{0.2cm}
  \item Download and install  \textbf{\color{NavyBlue}unzip} on the download image;\vspace{0.2cm}
  \item Download and install  \textbf{\color{NavyBlue}perl} on the download image;\vspace{0.2cm}
  \item Download and install  \textbf{\color{NavyBlue}FastQC} on the download image;\vspace{0.2cm}
  \item Run the embedded FastQC on fastq data.
\end{itemize}     
  }

\frame{
  \frametitle {Download and update a basic image}
  \begin{itemize}
  	\item Download Fedora; 
  	          	\begin{center}
  			\includegraphics[width=0.95\columnwidth]{./Figure/pullFedora}
  		\end{center}	
  		
  	 \end{itemize}	
  		}
  		
  		
 \frame{
  \frametitle {Download and update a basic image}
  \begin{itemize} 		
  	\item Create a new tag from Fedora image.
          	\begin{center}
  			\includegraphics[width=0.95\columnwidth]{./Figure/tagFastQC}
  			\end{center}	
 \end{itemize}
} 
  	 
 \frame{
  \frametitle {Download and update a basic image}
  \begin{itemize} 		
  	\item Update Fedora using \emph{\color{PineGreen} dnf update};
          	\begin{center}
  			\includegraphics[width=0.95\columnwidth]{./Figure/updateFedora}
  			\end{center}	
   \item  Commit the updated image.
   		    \begin{center}
  			\includegraphics[width=0.95\columnwidth]{./Figure/commitFedora}
  			\end{center}	
 \end{itemize}
} 
  	\frame{
  \frametitle {Download and install Oracle Java}
  \begin{itemize} 		
  	\item Download \textbf{\color{NavyBlue}Java RE ORACLE} (\url{http://www.oracle.com});\vspace{0.2cm}
  	\item Install  \textbf{\color{NavyBlue}Java RE ORACLE}
          	\begin{center}
  			\includegraphics[width=0.95\columnwidth]{./Figure/javaFedora}\vspace{0.2cm}
  			\end{center}	
   \item  Create a symbolic link in \emph{bin} for java program.
   		    \begin{center}
  			\includegraphics[width=0.95\columnwidth]{./Figure/javaFedora2}
  			\end{center}	
 \end{itemize}
}   	 
  
	\frame{
  \frametitle {Download and install FastQC}
  \begin{itemize} 		
  	\item Install \textbf{\color{NavyBlue} unzip} command using   \emph{\color{PineGreen} dnf install unzip.x86\_64};
          	\begin{center}
  			\includegraphics[width=0.95\columnwidth]{./Figure/unzipFedora}\vspace{0.2cm}
  			\end{center}		
  	\item Install  \textbf{\color{NavyBlue}perl} command using   \emph{\color{PineGreen} dnf install perl.x86\_64};
          	\begin{center}
  			\includegraphics[width=0.95\columnwidth]{./Figure/perlFedora}\vspace{0.2cm}
  			\end{center}	
   \end{itemize}
} 			
  
  			
   
\frame{
  \frametitle {Download and install FastQC}
  \begin{itemize}  			
   \item Unzip  FastQC program;
   		    \begin{center}
  			\includegraphics[width=0.95\columnwidth]{./Figure/fastqcFedora}
  			\end{center}	\vspace{0.1cm}
	\item Make FastQC program executable  and create a symbolic link in \emph{bin} for it;
	  	  		    \begin{center}
  			\includegraphics[width=0.95\columnwidth]{./Figure/fastqcFedora1}
  			\end{center}		\vspace{0.1cm}				
 \end{itemize}
}   	

\frame{
  \frametitle {Download and install FastQC}
  \begin{itemize}  			
 	\item Commit the updated image.
 	 	   \begin{center}
  			\includegraphics[width=0.95\columnwidth]{./Figure/fastqcFedora2}
  			\end{center}				
 \end{itemize}
}   	      
  	
 \frame{
  \frametitle { Download and install FastQC}
  \begin{itemize}  			
   \item  We use \textbf{\color{NavyBlue}gedit} to create our  bash script; \vspace{0.2cm}
   		    \begin{center}
  			\includegraphics[width=0.95\columnwidth]{./Figure/fastqcGedit}
  			\end{center}	
  \end{itemize}
}   	    	  

 \frame{
  \frametitle { Download and install FastQC}
  \begin{itemize}   	  			
	\item Update the image adding the created script. \end{itemize}
	\vspace{0.2cm}
	  	  		    \begin{center}
  			\includegraphics[width=1.0\columnwidth]{./Figure/fastqcshFedora}
  			\end{center}				

}  

 	    	  
 \frame{
  \frametitle {Run the embedded FastQC}
  \begin{itemize}   	  			
	\item Execute the embedded FastQC on .fastq files.\vspace{0.2cm}
	\item The folder containing .fastq files must be mount as /data/scratch  \end{itemize}
	
	  	    \begin{center}
  			\includegraphics[width=1.0\columnwidth]{./Figure/fastqrunFedora}
  			\end{center}	
  						

}

 	    	  
 \frame{
  \frametitle {Run the embedded FastQC}
  \begin{itemize}   	  			
	\item FastQC output:
	  	    \begin{center}
  			\includegraphics[width=0.78\columnwidth]{./Figure/fastqcHTML}
  			\end{center}	
  						
 \end{itemize}
}


   \frame{
  \frametitle {}
  \vspace{1.5cm}
  \centerline{\Huge \color{NavyBlue} \textbf{\emph{Create a docker image for }}}\vspace{0.05cm}
  \centerline{\Huge \color{NavyBlue} \textbf{\emph{FastQC tool using dockerfile}}}
  \vspace{0.4cm}
     	\begin{center}
  			\includegraphics[width=0.60\columnwidth]{./Figure/FastQC}
  		\end{center} 

}


          \frame{
  \frametitle {Create a dockerfile}
  We use \textbf{\color{NavyBlue}gedit} to create our dockerfile; \vspace{0.2cm}
 \begin{center}
  			\includegraphics[width=0.95\columnwidth]{./Figure/gedit2}
 \end{center} 
  }

 \frame{
  \frametitle {Create FastQC image with dockerfile}
  
 \emph{\color{PineGreen} docker build -t  ``beccuti/fastqc2'' . -f DockerfileFedora} can be used to build an image from a Dockerfile.
     	\begin{center}
  			\includegraphics[width=1.00\columnwidth]{./Figure/build2}
  		\end{center}  
  		     	\begin{center}
  			\includegraphics[width=1.00\columnwidth]{./Figure/build3}
  		\end{center}  
  
 } 	
  	

   \frame{
  \frametitle {}
  \vspace{1.5cm}
  \centerline{\Huge \color{NavyBlue} \textbf{\emph{A web server using docker}}}
  \vspace{0.4cm}
     	\begin{center}
  			\includegraphics[width=0.50\columnwidth]{./Figure/nginxDocker}
  		\end{center} 

}  

   \frame{
  \frametitle {A web server using docker}
	This requires the following tasks:\vspace{0.4cm}
	\begin{itemize}
	\item Download \textbf{\color{NavyBlue} nginx} docker image;\vspace{0.2cm}
	\item Execute this image using  the option \emph{\color{PineGreen}-p 80:80} to inform Docker that we want to expose the port 80. 
	\end{itemize}	  
  
    }
    
\frame{
\frametitle{A web server using docker}
    	\begin{center}
  			\includegraphics[width=0.50\columnwidth]{./Figure/nginx}
  		\end{center} 

\begin{itemize}
\item It is a HTTP server software with focus on core web server and proxy features;\vspace{0.2cm}
\item It was developed to provide   high concurrency, high performance and low memory usage;\vspace{0.2cm}
\item It is able to scale incredibly far with limited resources.
\end{itemize}
}  
   \frame{
  \frametitle {Download nginx docker image}
  \begin{itemize}
  	\item Download nginx image;   	 
  	\end{itemize}
  	          	\begin{center}
  			\includegraphics[width=1.0\columnwidth]{./Figure/pullnginx}
  		\end{center}	
  		
	
  		} 
   
 	     \frame{
\frametitle{Execute nginx docker image} 
  
 \emph{\color{PineGreen} docker run -p 80:80 nginx} must be used to star our web server on port 80.
     	\begin{center}
  			\includegraphics[width=1.00\columnwidth]{./Figure/runnginx}
  		\end{center}  
  	\begin{center}
  			\includegraphics[width=0.95\columnwidth]{./Figure/webnginx}
  		\end{center} 
 }  	
 
   \frame{
\frametitle{Execute nginx docker image}

\textbf{\color{NavyBlue}How to use nginx to visualize fastqc output:}
\vspace{0.4cm}
\begin{itemize}
\item Update nginx image with \textbf{\color{NavyBlue}vi} command;\vspace{0.2cm}
\item Modify   \textbf{\color{NavyBlue}/etc/nginx/conf.d/default.conf}  adding     \emph{\color{PineGreen}autoindex on;} in  \textbf{\color{NavyBlue}location /} block;\vspace{0.2cm}
\item Execute nginx mounting the folder containing fastqc output in \textbf{\color{NavyBlue}/local/shared/nginx/html}.
\end{itemize}  

    	\begin{center}
  			\includegraphics[width=0.50\columnwidth]{./Figure/nginx} \hspace{0.5cm} 
  			\includegraphics[width=0.15\columnwidth]{./Figure/QC}
  		\end{center} 
  		    
  	   }  
  
  
   \frame{
\frametitle{Execute nginx docker image} 	   
 \textbf{\color{NavyBlue}How to use nginx to visualize fastqc output:}
\vspace{0.4cm}
\begin{itemize}
\item Update nginx image with \textbf{\color{NavyBlue}vi} command;\vspace{0.2cm}
 \end{itemize} 	   
  	     	\begin{center}
  			\includegraphics[width=1.0\columnwidth]{./Figure/vimnginx}
  		\end{center}  
  		}
  
    \frame{
\frametitle{Execute nginx docker image} 	   
 \textbf{\color{NavyBlue}How to use nginx to visualize fastqc output:}
\vspace{0.4cm}
\begin{itemize}
\item Modify   \textbf{\color{NavyBlue}/etc/nginx/conf.d/default.conf}  adding     \emph{\color{PineGreen}autoindex on;} in  \textbf{\color{NavyBlue}location /} block;\vspace{0.2cm}
 \end{itemize} 	   
  	     	\begin{center}
  			\includegraphics[width=0.90\columnwidth]{./Figure/confnginx}
  		\end{center}  
  		} 
  		
      \frame{
\frametitle{Execute nginx docker image} 	   
 \textbf{\color{NavyBlue}How to use nginx to visualize fastqc output:}
\vspace{0.4cm}
\begin{itemize}
\item Commit the updates in the images;
 \end{itemize} 	   
  	     	\begin{center}
  			\includegraphics[width=0.90\columnwidth]{./Figure/commitnginx}
  		\end{center}  
  		} 
  		
  			
   		
      \frame{
\frametitle{Execute nginx docker image} 	   
 \textbf{\color{NavyBlue}How to use nginx to visualize fastqc output:}
\vspace{0.4cm}
\begin{itemize}
\item Run the new image \vspace{0.2cm}
\end{itemize}
  	     	\begin{center}
  			\includegraphics[width=1.0\columnwidth]{./Figure/run1ginx}
  		\end{center}
\begin{itemize} 		
\item Option \emph{\color{PineGreen}-v /home/beccuti/TMP/Pat/:/usr/share/nginx/html} is used to set the fastqc output folder as the root directory of web server;\vspace{0.2cm}
\item  Web server is executed in background  using \emph{\color{PineGreen} /usr/sbin/nginx -g "daemon off;"}.
 \end{itemize} 	   
  
  		} 
  		

  			
     \frame{
  \frametitle {}
  \vspace{1.5cm}
  \centerline{\Huge \color{NavyBlue} \textbf{\emph{Deploy services on a cluster }}}
  \centerline{\Huge \color{NavyBlue} \textbf{\emph{using docker swarm}}}
  \vspace{0.4cm}
     	\begin{center}
  			
  			\includegraphics[width=0.40\columnwidth]{./Figure/dockerswarm}
  		\end{center} 

}		 		
  

     \frame{
  \frametitle {Deploy services on a cluster }
  We will use  \emph{\color{NavyBlue} docker-machine} to create two virtual nodes. \vspace{0.4cm}
  \begin{itemize}
  \item It is a tool that lets you install Docker Engine on virtual hosts; \vspace{0.2cm}
  \item It requires an Hypervisor (e.g. Oracle VirtualBox) installed on the machine;\vspace{0.2cm}
  \end{itemize}
  	\begin{center}
  \includegraphics[width=0.30\columnwidth]{./Figure/dockermachine}
  		\end{center} 

}	  	


     \frame{
\frametitle{Deploy services on a cluster } 
  \vspace{0.1cm}
 \emph{\color{PineGreen} docker-machine create -\;-driver virtualbox myvm1} creates and starts a VirtualBox VM with Docker running.
     	\begin{center}
  			\includegraphics[width=1.00\columnwidth]{./Figure/createvm}
  		\end{center}  
 }  	  
  	

     \frame{
\frametitle{Deploy services on a cluster } 
  
 \emph{\color{PineGreen} docker-machine ls} returns the list of the created VirtualBox VMs.  \vspace{0.1cm}
     	\begin{center}
  			\includegraphics[width=0.90\columnwidth]{./Figure/lsvm}
  		\end{center}   
 \emph{\color{PineGreen} docker-machine stop myvm1} stops \textbf{\color{NavyBlue}myvm1} VM.  \vspace{-0.1cm}
     	\begin{center}
  			\includegraphics[width=0.90\columnwidth]{./Figure/stopvm}
  		\end{center} 
}

     \frame{
\frametitle{Deploy services on a cluster } 
    		 \emph{\color{PineGreen} docker-machine start myvm1} starts \textbf{\color{NavyBlue}myvm1} VM.  \vspace{0.3cm}
     	\begin{center}
  			\includegraphics[width=0.90\columnwidth]{./Figure/runvm}
  		\end{center}  
 }  	    	
  
  
   \frame{
\frametitle{Deploy services on a cluster } 
  
 \emph{\color{PineGreen} docker-machine rm myvm1} removes \textbf{\color{NavyBlue}myvm1} VM.  \vspace{0.1cm}
     	\begin{center}
  			\includegraphics[width=0.90\columnwidth]{./Figure/rmvm}
  		\end{center}

 } 
 
    \frame{
\frametitle{Deploy services on a cluster } 
  
   		 \emph{\color{PineGreen} docker-machine ssh myvm1} can be used to access  \textbf{\color{NavyBlue}myvm1} VM.  \vspace{0.2cm}
     	\begin{center}
  			\includegraphics[width=1.0\columnwidth]{./Figure/sshvm}
  		\end{center}    
}
 
   
   \frame{
\frametitle{Deploy services on a cluster } 
   	    	  
   \emph{\color{PineGreen} docker-machine ssh myvm1 "docker ps -a"} can be used to execute commands on \textbf{\color{NavyBlue}myvm1} VM.  \vspace{0.2cm}
     	\begin{center}
  			\includegraphics[width=1.00\columnwidth]{./Figure/execvm}
  		\end{center} 
  		}

   \frame{
\frametitle{Deploy services on a cluster }  
\emph{\color{PineGreen} docker-machine ssh  myvm1 "docker swarm init --advertise-addr 192.168.99.102"}  is used to initialize the swarm and set \textbf{\color{NavyBlue}myvm1} as manager.
     	\begin{center}
  			\includegraphics[width=1.00\columnwidth]{./Figure/initswarm}
  		\end{center} 
\emph{\color{PineGreen} docker-machine ssh  myvm2 "docker swarm join --token SWMTKN-1-430vpzpm8ves6whcy8ox34vlyrrmutinoewrzcaaqix32h917f-7bml90kfo5j1toxmql19hfsmh 192.168.99.102:2377"}  is used to add \textbf{\color{NavyBlue}myvm2} into the  swarm.
     	\begin{center}
  			\includegraphics[width=1.00\columnwidth]{./Figure/addswarm}
  		\end{center} 

}
 
 
    \frame{
\frametitle{}
 \centerline{\Huge \color{NavyBlue} \textbf{\emph{ Deploy services}} }\vspace{0.1cm} \centerline{\Huge \color{NavyBlue} \textbf{\emph{into a Docker swarm}}}
     	\begin{center}
  			
  			\includegraphics[width=0.40\columnwidth]{./Figure/dockerswarm}
  		\end{center} 
} 
  
   \frame{
\frametitle{Deploy services on a cluster } 
   	    	  
   \emph{\color{PineGreen} docker-machine ssh myvm1 "docker node ls"} provides information about the swarm nodes.  \vspace{-0.1cm}
     	\begin{center}
  			\includegraphics[width=1.0\columnwidth]{./Figure/nodelsswarm}
  		\end{center} 		
   \emph{\color{PineGreen}docker-machine ssh  myvm1 "docker node update -\;-availability drain  myvm2"} makes a node inactive.  \vspace{-0.1cm}
     	\begin{center}
  			\includegraphics[width=1.0\columnwidth]{./Figure/nodeinacswarm}
  		\end{center} 
  		}   		
  	
   \frame{
\frametitle{Deploy services on a cluster } 
   	    	
   \emph{\color{PineGreen}docker-machine ssh  myvm1 "docker node update -\;-availability active  myvm2"} makes a node inactive.  \vspace{0.2cm}
     	\begin{center}
  			\includegraphics[width=1.0\columnwidth]{./Figure/nodeactswarm}
  		\end{center} 
  		}   	  	
  		
  	  		
  		
  	   \frame{
\frametitle{Deploy services on a cluster } 
   	    	  
   \emph{\color{PineGreen} docker-machine ssh myvm1 "docker service create   -\;-name my-web   -\;-publish published=80,target=80 -\;-replicas=1     nginx"} starts one service on a node.  \vspace{-0.1cm}
     	\begin{center}
  			\includegraphics[width=1.00\columnwidth]{./Figure/serviceswarm}
  		\end{center} 	
  				
  \emph{\color{PineGreen}docker-machine ssh  myvm1 "docker  service ls"} and \emph{\color{PineGreen}docker-machine ssh  myvm1 "docker service  ps my-web"}
    return information on services.
    
       	\begin{center}
  			\includegraphics[width=1.00\columnwidth]{./Figure/serviceinfoswarm}
  		\end{center}  
  		}	
    		
  	   \frame{
\frametitle{Deploy services on a cluster } 
   	    	  
   \emph{\color{PineGreen}  docker-machine ssh  myvm1 "docker service scale my-web=2" 	} can be used to scale a service.  \vspace{-0.1cm}
     	\begin{center}
  			\includegraphics[width=1.0\columnwidth]{./Figure/scaleswarm}
  		\end{center} 	

       	\begin{center}
  			\includegraphics[width=1.0\columnwidth]{./Figure/serviceinfoswarm2}
  		\end{center}  
  		}		
  		
  	 	   \frame{
\frametitle{Deploy services on a cluster } 
   	    	  
   \emph{\color{PineGreen} docker-machine ssh  myvm1 "docker service create -\;-name my-web   -\;-publish published=80,target=80   - -mode global  nginx"
 	} can be use to automatically allocate a service on all the nodes  \vspace{-0.1cm}
     	\begin{center}
  			\includegraphics[width=1.0\columnwidth]{./Figure/serviceglobalswarm}
  		\end{center} 	

       	\begin{center}
  			\includegraphics[width=1.0\columnwidth]{./Figure/serviceglobalswarm2}
  		\end{center}  
  		}		
	
    		
  	 	   \frame{
\frametitle{Deploy services on a cluster } 
\textbf{\color{NavyBlue}How to execute a one-shot service using swam}\vspace{0.4cm}
\begin{itemize}
\item Our goal is to use fastqc image\footnote{It must be uploaded in Docker hub} in parallel on the two nodes; \vspace{0.2cm}
\item Data can be shared using Virtualbox shared folder (i.e. \textbf{\color{NavyBlue}/hosthome}); \vspace{0.2cm}
\item We create two folders with containing fastq files. (i.e. \textbf{\color{NavyBlue}d1/} and \textbf{\color{NavyBlue}d2/});\vspace{0.2cm}
\item Option  \emph{\color{PineGreen} -\;-restart-condition=``none" } can be used for running  one-shot service;\vspace{0.2cm}
\item Option  \emph{\color{PineGreen} --mount type=bind,src=$\langle SOURCE \rangle$ ,dst= $\langle DESTINATION \rangle$} can be used for mounting the input data folder;
\end{itemize}

}		

  	 	   \frame{
\frametitle{Deploy services on a cluster } 
\textbf{\color{NavyBlue}How to execute a one-shot service using swam}\\\vspace{0.4cm}
\emph{\color{PineGreen}docker-machine  ssh   myvm1\\
\hspace{0.5cm}"docker service create -\;-replicas 1 -\;-name S1 -\;-user 1000  \\
\hspace{0.5cm}-\;-restart-condition="none"   \\
\hspace{0.5cm}-\;-mount type=bind,\\
\hspace{0.5cm}src=/hosthome/beccuti/Articoli/Presentation/DockerCorso/data/d2,\\
\hspace{0.5cm}dst=/data/scratch \\
\hspace{0.5cm}docker.io/beccuti/fastqc2   /bin/fastqc.sh"}

       	\begin{center}
  			\includegraphics[width=1.0\columnwidth]{./Figure/serviceoneswarm}
  		\end{center} 

} 


 	 	   \frame{
\frametitle{Deploy services on a cluster } 
\textbf{\color{NavyBlue}How to execute a one-shot service using swam}\\\vspace{0.4cm}
\emph{\color{PineGreen}docker-machine  ssh   myvm1\\
\hspace{0.5cm}"docker service create -\;-replicas 1 -\;-name S2 -\;-user 1000  \\
\hspace{0.5cm}-\;-restart-condition="none"   \\
\hspace{0.5cm}-\;-mount type=bind,\\
\hspace{0.5cm}src=/hosthome/beccuti/Articoli/Presentation/DockerCorso/data/d1,\\
\hspace{0.5cm}dst=/data/scratch \\
\hspace{0.5cm}docker.io/beccuti/fastqc2   /bin/fastqc.sh"}

       	\begin{center}
  			\includegraphics[width=1.0\columnwidth]{./Figure/serviceoneswarm1}
  		\end{center} 

} 


  	 	   \frame{
\frametitle{Deploy services on a cluster } 
\textbf{\color{NavyBlue}How to execute a one-shot service using swam}\vspace{0.4cm}
       	\begin{center}
  			\includegraphics[width=1.0\columnwidth]{./Figure/serviceoneps}
  		\end{center} 
  		       	\begin{center}
  			\includegraphics[width=1.0\columnwidth]{./Figure/serviceoneps1}
  		\end{center} 
}
 		

  		\frame{
\frametitle{Conclusion}
In this training day:\vspace{0.1cm}
\begin{enumerate}
\item A short introduction  recalling the concepts described in the first day\vspace{0.1cm}
\begin{itemize}
\item Virtualization: Virtual Machines and containers; \vspace{0.1cm}
\item Containers in Linux: Docker project; \vspace{0.1cm}
\end{itemize}

\item A simple example: how to embed an application  in docker image; \vspace{0.1cm}
\item Create a docker image for FastQC tool;\vspace{0.1cm}
\item A web server  using docker;\vspace{0.1cm}
\item Deploy services on a cluster using docker swarm.
\end{enumerate}
\centerline{\includegraphics[scale=0.22]{./Figure/end}}
}
 		
 	
\frame{
\frametitle{}
\vspace{2.0cm}
\centerline{\includegraphics[scale=0.55]{./Figure/thanks.jpg}}
}


\frame{
\frametitle {}
\vspace{2cm}
\centerline{\Huge \color{NavyBlue} \textbf{\emph{Storage}}}
\vspace{0.5cm}
}
\include{storage}



\frame{
\frametitle {}
\vspace{2cm}
\centerline{\Huge \color{NavyBlue} \textbf{\emph{Swarm}}}
\vspace{0.5cm}
}
\begin{frame}
\frametitle{Swarm}
\framesubtitle{Overview}
Multiple Docker hosts can be managed in a centralized manner using an inner feature of Docker: \textbf{\textit{Swarm}}
\end{frame}

\begin{frame}
\frametitle{Swarm}
\framesubtitle{Overview}
\begin{itemize}
\item Cluster management
\item Decentralized design (one is all)
\item Declarative service model
\item Scaling
\item Desired state reconciliation
\item Multi-host networking
\item Service discovery
\item Load balancing
\item Secure by default (TLS)
\item Rolling updates
\end{itemize}
\end{frame}

\begin{frame}
\frametitle{Swarm}
\framesubtitle{Overview}
A generic Docker Host is a \textit{Node} in the Swarm.
\end{frame}

\begin{frame}
\frametitle{Swarm}
\framesubtitle{Overview}
A generic Docker Host is a \textit{Node} in the Swarm.\\
\vspace{0.4cm}
A Node can be:
\begin{itemize}
\item Manger: dispatches tasks to workers; orchestrates and manages the cluster
\item Worker: executes a task  
\end{itemize}
\end{frame}

\begin{frame}
\frametitle{Swarm}
\framesubtitle{Overview}
A \textit{Service} is the formal definition of a Task.\\
\vspace{0.4cm}
A \textit{Task} is the instance of a Service.
\end{frame}

\begin{frame}
\frametitle{Swarm}
\framesubtitle{Overview}
\begin{center}
\includegraphics[width=\columnwidth]{./Figure/swarm-diagram}
\end{center}
\end{frame}

\begin{frame}
\frametitle{Swarm}
\framesubtitle{Setup}
You can create a Swarm environment using \href{https://www.virtualbox.org/wiki/Downloads}{VirtualBox}
\vspace{0.4cm}
\begin{itemize}
\item \href{https://download.virtualbox.org/virtualbox/6.0.8/VirtualBox-6.0.8-130520-Win.exe}{Windows}
\item \href{https://download.virtualbox.org/virtualbox/6.0.8/VirtualBox-6.0.8-130520-OSX.dmg}{OS X}
\item \href{https://www.virtualbox.org/wiki/Linux_Downloads}{Lilnux}
  \begin{itemize}
    \item \href{https://download.virtualbox.org/virtualbox/6.0.8/virtualbox-6.0_6.0.8-130520~Ubuntu~bionic_amd64.deb}{Ubuntu}
    \item \href{http://dl-cdn.alpinelinux.org/alpine/v3.9/releases/x86_64/alpine-virt-3.9.4-x86_64.iso}{AlpineLinux}
  \end{itemize}
\end{itemize}
\end{frame}


\begin{frame}[fragile]
\frametitle{Swarm}
\framesubtitle{Setup: VirtualBox}
\begin{lstlisting}
VBoxManage natnetwork add 
  --netname swarm 
  --network "192.168.15.0/24" 
  --enable --dhcp on
\end{lstlisting}
\tiny
\href{https://www.virtualbox.org/manual/ch06.html#network_nat_service}{VirtualBox NAT service}
\normalsize
\end{frame}


\begin{frame}
\frametitle{Swarm}
\framesubtitle{Setup: Virtual Machine}
Create a virtual machine with default parameters
\end{frame}

\begin{frame}[fragile]
\frametitle{Swarm}
\framesubtitle{Setup: Initialize the Virtual Machine}
Attach \lstinline!Adapter 1! to \lstinline!NAT Network! and select the name \lstinline!swarm!\\
\begin{center}
\includegraphics[width=\columnwidth]{./Figure/swarm-vm-init-net}
\end{center}
\end{frame}


\begin{frame}[fragile]
\frametitle{Swarm}
\framesubtitle{Setup: Initialize the Virtual Machine}
Attach to the \lstinline!Storage! cd-rom the \href{http://dl-cdn.alpinelinux.org/alpine/v3.9/releases/x86_64/alpine-virt-3.9.4-x86_64.iso}{AlpineLinux} iso. \\
\begin{center}
\includegraphics[width=\columnwidth]{./Figure/swarm-vm-init-cdrom}
\end{center}
\end{frame}

\begin{frame}[fragile]
\frametitle{Swarm}
\framesubtitle{Setup: Install AlpineLinux}
\begin{itemize}
\item Start the vm 
\item proceed with the installation procedure typing \lstinline!setup-alpine!
\item Set the network to \lstinline!DHCP!\\
\item Select the disk \lstinline!sda!
\item Select request mode \lstinline!sys!
\item \lstinline!reboot!
\end{itemize}
\end{frame}


\begin{frame}[fragile]
\frametitle{Swarm}
\framesubtitle{Setup: Install Docker into AlpineLinux}
\begin{itemize}
\item Start the vm
\item Login
\item 
\tiny
\begin{lstlisting}
echo"http://dl-cdn.alpinelinux.org/alpine/latest-stable/community" >> /etc/apk/repositories
\end{lstlisting}
\normalsize
\item \lstinline!apk update!
\item \lstinline!apk add docker!
\item \lstinline!rc-update add docker boot!
\item \lstinline!reboot!
\end{itemize}
\end{frame}


\begin{frame}[fragile]
\frametitle{Swarm}
\framesubtitle{Setup: Virtual Machines}
Do a full clone of the first machine creating 2 more.\\
\vspace{0.4cm}
For each clone regenerate the MAC address\\
\vspace{0.4cm}
Once logged into the machine change the hostname and reboot\\
\begin{lstlisting}
echo "SwarmNode1" > /etc/hostname
reboot
\end{lstlisting}

\begin{lstlisting}
echo "SwarmNode2" > /etc/hostname
reboot
\end{lstlisting}
\end{frame}


\begin{frame}[fragile]
\frametitle{Swarm}
\framesubtitle{Setup: Virtual Machines}
In every VM print the IP addess assigned by the DHCP. \\
\lstinline!ifconfig! \\
We need them to initialize Docker Swarm
\end{frame}

\frame{
\frametitle {}
\vspace{2cm}
\centerline{\Huge \color{NavyBlue} \textbf{\emph{Swarm}}}
\vspace{0.5cm}
}

     \frame{
  \frametitle {Deploy services on a cluster }
  We will use  \emph{\color{NavyBlue} docker-machine} to create two virtual nodes. \vspace{0.4cm}
  \begin{itemize}
  \item It is a tool that lets you install Docker Engine on virtual hosts; \vspace{0.2cm}
  \item It requires an Hypervisor (e.g. Oracle VirtualBox) installed on the machine;\vspace{0.2cm}
  \end{itemize}
  	\begin{center}
  \includegraphics[width=0.30\columnwidth]{./Figure/dockermachine}
  		\end{center} 

}	  	


     \frame{
\frametitle{Deploy services on a cluster } 
  \vspace{0.1cm}
 \emph{\color{PineGreen} docker-machine create -\;-driver virtualbox myvm1} creates and starts a VirtualBox VM with Docker running.
     	\begin{center}
  			\includegraphics[width=1.00\columnwidth]{./Figure/createvm}
  		\end{center}  
 }  	  
  	

     \frame{
\frametitle{Deploy services on a cluster } 
  
 \emph{\color{PineGreen} docker-machine ls} returns the list of the created VirtualBox VMs.  \vspace{0.1cm}
     	\begin{center}
  			\includegraphics[width=0.90\columnwidth]{./Figure/lsvm}
  		\end{center}   
 \emph{\color{PineGreen} docker-machine stop myvm1} stops \textbf{\color{NavyBlue}myvm1} VM.  \vspace{-0.1cm}
     	\begin{center}
  			\includegraphics[width=0.90\columnwidth]{./Figure/stopvm}
  		\end{center} 
}

     \frame{
\frametitle{Deploy services on a cluster } 
    		 \emph{\color{PineGreen} docker-machine start myvm1} starts \textbf{\color{NavyBlue}myvm1} VM.  \vspace{0.3cm}
     	\begin{center}
  			\includegraphics[width=0.90\columnwidth]{./Figure/runvm}
  		\end{center}  
 }  	    	
  
  
   \frame{
\frametitle{Deploy services on a cluster } 
  
 \emph{\color{PineGreen} docker-machine rm myvm1} removes \textbf{\color{NavyBlue}myvm1} VM.  \vspace{0.1cm}
     	\begin{center}
  			\includegraphics[width=0.90\columnwidth]{./Figure/rmvm}
  		\end{center}

 } 
 
    \frame{
\frametitle{Deploy services on a cluster } 
  
   		 \emph{\color{PineGreen} docker-machine ssh myvm1} can be used to access  \textbf{\color{NavyBlue}myvm1} VM.  \vspace{0.2cm}
     	\begin{center}
  			\includegraphics[width=1.0\columnwidth]{./Figure/sshvm}
  		\end{center}    
}
 
   
   \frame{
\frametitle{Deploy services on a cluster } 
   	    	  
   \emph{\color{PineGreen} docker-machine ssh myvm1 "docker ps -a"} can be used to execute commands on \textbf{\color{NavyBlue}myvm1} VM.  \vspace{0.2cm}
     	\begin{center}
  			\includegraphics[width=1.00\columnwidth]{./Figure/execvm}
  		\end{center} 
  		}

   \frame{
\frametitle{Deploy services on a cluster }  
\emph{\color{PineGreen} docker-machine ssh  myvm1 "docker swarm init --advertise-addr 192.168.99.102"}  is used to initialize the swarm and set \textbf{\color{NavyBlue}myvm1} as manager.
     	\begin{center}
  			\includegraphics[width=1.00\columnwidth]{./Figure/initswarm}
  		\end{center} 
\emph{\color{PineGreen} docker-machine ssh  myvm2 "docker swarm join --token SWMTKN-1-430vpzpm8ves6whcy8ox34vlyrrmutinoewrzcaaqix32h917f-7bml90kfo5j1toxmql19hfsmh 192.168.99.102:2377"}  is used to add \textbf{\color{NavyBlue}myvm2} into the  swarm.
     	\begin{center}
  			\includegraphics[width=1.00\columnwidth]{./Figure/addswarm}
  		\end{center} 

}
 
 
    \frame{
\frametitle{}
 \centerline{\Huge \color{NavyBlue} \textbf{\emph{ Deploy services}} }\vspace{0.1cm} \centerline{\Huge \color{NavyBlue} \textbf{\emph{into a Docker swarm}}}
     	\begin{center}
  			
  			\includegraphics[width=0.40\columnwidth]{./Figure/dockerswarm}
  		\end{center} 
} 
  
   \frame{
\frametitle{Deploy services on a cluster } 
   	    	  
   \emph{\color{PineGreen} docker-machine ssh myvm1 "docker node ls"} provides information about the swarm nodes.  \vspace{-0.1cm}
     	\begin{center}
  			\includegraphics[width=1.0\columnwidth]{./Figure/nodelsswarm}
  		\end{center} 		
   \emph{\color{PineGreen}docker-machine ssh  myvm1 "docker node update -\;-availability drain  myvm2"} makes a node inactive.  \vspace{-0.1cm}
     	\begin{center}
  			\includegraphics[width=1.0\columnwidth]{./Figure/nodeinacswarm}
  		\end{center} 
  		}   		
  	
   \frame{
\frametitle{Deploy services on a cluster } 
   	    	
   \emph{\color{PineGreen}docker-machine ssh  myvm1 "docker node update -\;-availability active  myvm2"} makes a node inactive.  \vspace{0.2cm}
     	\begin{center}
  			\includegraphics[width=1.0\columnwidth]{./Figure/nodeactswarm}
  		\end{center} 
  		}   	  	
  		
  	  		
  		
  	   \frame{
\frametitle{Deploy services on a cluster } 
   	    	  
   \emph{\color{PineGreen} docker-machine ssh myvm1 "docker service create   -\;-name my-web   -\;-publish published=80,target=80 -\;-replicas=1     nginx"} starts one service on a node.  \vspace{-0.1cm}
     	\begin{center}
  			\includegraphics[width=1.00\columnwidth]{./Figure/serviceswarm}
  		\end{center} 	
  				
  \emph{\color{PineGreen}docker-machine ssh  myvm1 "docker  service ls"} and \emph{\color{PineGreen}docker-machine ssh  myvm1 "docker service  ps my-web"}
    return information on services.
    
       	\begin{center}
  			\includegraphics[width=1.00\columnwidth]{./Figure/serviceinfoswarm}
  		\end{center}  
  		}	
    		
  	   \frame{
\frametitle{Deploy services on a cluster } 
   	    	  
   \emph{\color{PineGreen}  docker-machine ssh  myvm1 "docker service scale my-web=2" 	} can be used to scale a service.  \vspace{-0.1cm}
     	\begin{center}
  			\includegraphics[width=1.0\columnwidth]{./Figure/scaleswarm}
  		\end{center} 	

       	\begin{center}
  			\includegraphics[width=1.0\columnwidth]{./Figure/serviceinfoswarm2}
  		\end{center}  
  		}		
  		
  	 	   \frame{
\frametitle{Deploy services on a cluster } 
   	    	  
   \emph{\color{PineGreen} docker-machine ssh  myvm1 "docker service create -\;-name my-web   -\;-publish published=80,target=80   - -mode global  nginx"
 	} can be use to automatically allocate a service on all the nodes  \vspace{-0.1cm}
     	\begin{center}
  			\includegraphics[width=1.0\columnwidth]{./Figure/serviceglobalswarm}
  		\end{center} 	

       	\begin{center}
  			\includegraphics[width=1.0\columnwidth]{./Figure/serviceglobalswarm2}
  		\end{center}  
  		}		
	
    		
  	 	   \frame{
\frametitle{Deploy services on a cluster } 
\textbf{\color{NavyBlue}How to execute a one-shot service using swam}\vspace{0.4cm}
\begin{itemize}
\item Our goal is to use fastqc image\footnote{It must be uploaded in Docker hub} in parallel on the two nodes; \vspace{0.2cm}
\item Data can be shared using Virtualbox shared folder (i.e. \textbf{\color{NavyBlue}/hosthome}); \vspace{0.2cm}
\item We create two folders with containing fastq files. (i.e. \textbf{\color{NavyBlue}d1/} and \textbf{\color{NavyBlue}d2/});\vspace{0.2cm}
\item Option  \emph{\color{PineGreen} -\;-restart-condition=``none" } can be used for running  one-shot service;\vspace{0.2cm}
\item Option  \emph{\color{PineGreen} --mount type=bind,src=$\langle SOURCE \rangle$ ,dst= $\langle DESTINATION \rangle$} can be used for mounting the input data folder;
\end{itemize}

}		

  	 	   \frame{
\frametitle{Deploy services on a cluster } 
\textbf{\color{NavyBlue}How to execute a one-shot service using swam}\\\vspace{0.4cm}
\emph{\color{PineGreen}docker-machine  ssh   myvm1\\
\hspace{0.5cm}"docker service create -\;-replicas 1 -\;-name S1 -\;-user 1000  \\
\hspace{0.5cm}-\;-restart-condition="none"   \\
\hspace{0.5cm}-\;-mount type=bind,\\
\hspace{0.5cm}src=/hosthome/beccuti/Articoli/Presentation/DockerCorso/data/d2,\\
\hspace{0.5cm}dst=/data/scratch \\
\hspace{0.5cm}docker.io/beccuti/fastqc2   /bin/fastqc.sh"}

       	\begin{center}
  			\includegraphics[width=1.0\columnwidth]{./Figure/serviceoneswarm}
  		\end{center} 

} 


 	 	   \frame{
\frametitle{Deploy services on a cluster } 
\textbf{\color{NavyBlue}How to execute a one-shot service using swam}\\\vspace{0.4cm}
\emph{\color{PineGreen}docker-machine  ssh   myvm1\\
\hspace{0.5cm}"docker service create -\;-replicas 1 -\;-name S2 -\;-user 1000  \\
\hspace{0.5cm}-\;-restart-condition="none"   \\
\hspace{0.5cm}-\;-mount type=bind,\\
\hspace{0.5cm}src=/hosthome/beccuti/Articoli/Presentation/DockerCorso/data/d1,\\
\hspace{0.5cm}dst=/data/scratch \\
\hspace{0.5cm}docker.io/beccuti/fastqc2   /bin/fastqc.sh"}

       	\begin{center}
  			\includegraphics[width=1.0\columnwidth]{./Figure/serviceoneswarm1}
  		\end{center} 

} 


  	 	   \frame{
\frametitle{Deploy services on a cluster } 
\textbf{\color{NavyBlue}How to execute a one-shot service using swam}\vspace{0.4cm}
       	\begin{center}
  			\includegraphics[width=1.0\columnwidth]{./Figure/serviceoneps}
  		\end{center} 
  		       	\begin{center}
  			\includegraphics[width=1.0\columnwidth]{./Figure/serviceoneps1}
  		\end{center} 
}
 		


\frame{
\frametitle {}
\vspace{2cm}
\centerline{\Huge \color{NavyBlue} \textbf{\emph{Private Registry}}}
\vspace{0.5cm}
}


\begin{frame}
\frametitle{Private registry}
A local Docker registry can be useful in many situations
\begin{itemize}
\item images contain private data or information
\item need to test specific applications
\item speed and reliability
\item other applications require the service
\end{itemize}
\end{frame}


\begin{frame}
\frametitle{Private registry}
A local Docker registry can be useful in many situations
\begin{itemize}
\item images contain private data or information
  \begin{itemize}
  \item passwords
  \item user names
  \item network configurations
  \item mount points
  \end{itemize}
\item need to test specific applications
\item speed and reliability
\item other applications require the service
\end{itemize}
\end{frame}


\begin{frame}
\frametitle{Private registry}

A local Docker registry can be useful in many situations

\begin{itemize}
\item images contain private data or information
\item need to test specific applications
\item speed and reliability
  \begin{itemize}
  \item good internet connection
  \item not limited by number of images or containers
  \item lots of disk space
  \end{itemize}
\item other applications require the service
\end{itemize}
\end{frame}


\begin{frame}
\frametitle{Private registry}

A local Docker registry can be useful in many situations

\begin{itemize}
\item images contain private data or information
\item need to test specific applications
\item speed and reliability
\item other applications require the service
  \begin{itemize}
  \item workflow managers may use Docker container for running the pipelines
  \item other container technologies depends on custom Docker images
  \end{itemize}
\end{itemize}
\end{frame}


\begin{frame}[fragile]
\frametitle{Docker Registry}
\framesubtitle{run}
Run an insecure registry

\begin{lstlisting}
$ docker run -d -p 5000:5000 \
           --restart=always \
		   --name registry registry:2
\end{lstlisting}

\textit{!!! WARNING !!!} this is an insecure registry.
\end{frame}

\begin{frame}[fragile]
\frametitle{Docker Registry}
\framesubtitle{run}
Run an insecure registry

\begin{lstlisting}
$ docker run -d -p 5000:5000 \
           --restart=always \
		   --name registry registry:2
\end{lstlisting}

This registry runs on the localhost

and
\end{frame}

\begin{frame}[fragile]
\frametitle{Docker Registry}
\framesubtitle{run}
Run an insecure registry

\begin{lstlisting}
$ docker run -d -p 5000:5000 \
           --restart=always \
		   --name registry registry:2
\end{lstlisting}
This registry runs on the localhost

and

is \textit{INSECURE} but it's OK for testing.
\end{frame}

\begin{frame}[fragile]
\frametitle{Docker Registry}
\framesubtitle{run}

Check if the registry is running

\begin{lstlisting}
$docker ps
CONTAINER ID  IMAGE      COMMAND
d37dd351dd30  registry:2 "/entrypoint.sh /e..."

CREATED    STATUS    PORTS                  
15 min ago Up 15 min 0.0.0.0:5000->5000/tcp 

NAMES
registry
\end{lstlisting}
\end{frame}

\begin{frame}[fragile]
\frametitle{Docker Registry}
\framesubtitle{load an image}

Get an image from the net

\begin{lstlisting}
$ docker pull ubuntu:18.04
\end{lstlisting}
\end{frame}

\begin{frame}[fragile]
\frametitle{Docker Registry}
\framesubtitle{load an image}

Tag the image with a proper name
\begin{lstlisting}
$ docker tag ubuntu:18.04 \
    localhost:5000/user/mydistro:18.04
    
\end{lstlisting}
\end{frame}

\begin{frame}[fragile]
\frametitle{Docker Registry}
\framesubtitle{load an image}

Push the image to the local repository

\begin{lstlisting}
$ docker push \
  localhost:5000/user/mydistro:18.04
\end{lstlisting}
\end{frame}

\begin{frame}[fragile]
\frametitle{Docker Registry}
\framesubtitle{list images}

Docker works with HTTP API (v2)

\begin{lstlisting}
$ curl -v http://localhost:5000/v2/_catalog
\end{lstlisting}

\url{https://docs.docker.com/registry/spec/api/}

\end{frame}

\begin{frame}[fragile]
\frametitle{Docker Registry}
\framesubtitle{list images}

\begin{lstlisting}
< HTTP/1.1 200 OK
< Content-Type: application/json; charset=utf-8
< Docker-Distribution-Api-Version: registry/2.0
< X-Content-Type-Options: nosniff
< Date: Tue, 25 Sep 2018 13:36:04 GMT
< Content-Length: 52
<
{"repositories":["joe/ubuntu","raoul/ubuntu"]}
* Connection #0 to host localhost left intact
\end{lstlisting}
\end{frame}

\begin{frame}[fragile]
\frametitle{Docker Registry}
\framesubtitle{get tags}

\begin{lstlisting}
$ curl http://localhost:5000/raoul/ubuntu/tags/list
{"name":"raoul/ubuntu","tags":["18.04"]}
\end{lstlisting}
\end{frame}

\begin{frame}[fragile]
\frametitle{Docker Registry}
\framesubtitle{get details}

\begin{lstlisting}
$ curl http://localhost:5000/raoul/ubuntu/manifests/18.04
\end{lstlisting}

\includegraphics[width=0.8\columnwidth]{./Figure/RegistryDetails}
\end{frame}

\begin{frame}[fragile]
\frametitle{Docker Hub}
\framesubtitle{public registry}

\includegraphics[width=0.8\columnwidth]{./Figure/docker-222-114}
\end{frame}


\begin{frame}[fragile]
\frametitle{Docker Hub}
\framesubtitle{public registry}

\includegraphics[width=0.8\columnwidth]{./Figure/docker-223-116}
\end{frame}

\begin{frame}[fragile]
\frametitle{Docker Hub}
\framesubtitle{public registry}
\begin{itemize}
\item register a new user to \url{https://hub.docker.com/}
\item \lstinline!export DOCKER_ID_USER="username"!
\item \lstinline!docker login!
\item \lstinline!docker tag imageX $DOCKER_ID_USER/imageX!
\item \lstinline!docker push $DOCKER_ID_USER/imageX!
\end{itemize}
\end{frame}
					   
\end{document}


